\documentclass[
    a4paper,
    doc,
    12pt,
    natbib,
]{apa6}

\usepackage[german]{babel}
\usepackage[utf8x]{inputenc}
\usepackage{amsmath}
\usepackage{graphicx}
\usepackage{mathptmx} % Für Times New Roman Font
\usepackage[colorinlistoftodos]{todonotes}

\setlength {\marginparwidth }{2cm} 

\title{Exposé: Analyse der Wirksamkeit  der Spielelemente Abzeichen, Fortschrittsanzeige und Rangliste hinsichtlich Motivation und Leistung im Kontext der Kommandozeile}
\shorttitle{Exposé zur Bachelor-Thesis}
\author{Leon Morten Richter\\MatrikelNr.: 1105170}
\affiliation{Christian-Albrechts-Universität zu Kiel}


\begin{document}
\maketitle

\section{Motivation}
Gamification ist definiert als der Einsatz spieltypischer Elemente in einem spielfremden Kontext \citep{deterding_gamification_2011} und zielt in aller Regel auf eine Motivationssteigerung ab \citep{noauthor_gamification_2010}. Anwendungsgebiete, in denen Gamification erfolgreich eingesetzt werden kann, sind Forschung \citep{brauer_erhohung_2019}, Gesundheit und Fitness \citep{johnson_gamification_2016} und Bildung \citep{de_freitas_prepared_2006}. Insbesondere im Bildungsbereich konnte gezeigt werden, dass Gamification einen positiven Einfluss auf Motivation und Leistung haben kann \citep{ibanez_gamification_2014}, da der Lernerfolg wesentlich durch Motivation, Interesse und Engagement der Lernenden bedingt ist \citep{astin_student_1984}. Allerdings ist Gamification kein Allheilmittel. Zunächst erfordert das Design und die Umsetzung von Spielelementen Arbeit und stellt damit einen Mehraufwand für Lehrende dar. Gleichzeitig konnte gezeigt werden, dass Gamification eine Reduzierung der Leistung, unerwünschte Seiteneffekte, Ineffizienz und sogar einen Motivationsverlust zur Folge haben kann \citep{toda_dark_2018}. Aus diesem Grund ist die jeweilige Umsetzung von Gamification auf ihren Nutzen hin zu untersuchen und zu bewerten. Ein solches Anwendungsfeld ist die Kommandozeile. Die Kommandozeile ist zentrales Element moderner Betriebssysteme und für die Softwareentwicklung und Administration unerlässlich. Da die Kommandozeile eine rein textbasierte Schnittstelle bietet, stellt diese für viele Einsteiger eine neue Art der Mensch-Maschine-Interaktion dar. Das Erlernen eben dieser kann anfänglich mit einem hohen Maß an Frustration verbunden sein. So ist es wenig verwunderlich, dass es eine Vielzahl unterschiedlicher Spiele gibt, deren Ziel darin besteht, die Grundlagen der Kommandozeile spielend zu vermitteln. Diese Art der Spiele wird als Serious Game bezeichnet. Damit sind Spiele gemeint, deren Zweck nicht ausschließlich in der Unterhaltung besteht \citep{djaouti_classifying_2011} und die beispielsweise ein Lernziel verfolgen. Eine kurze Suche\footnote{Link: https://github.com/topics/terminal-game} nach dem Begriff \glqq terminal-game\grqq{} auf der Code-Hosting-Plattform GitHub bringt (Stand Mai 2020) über 300 Ergebnisse hervor. Trotz der offensichtlichen Popularität wurde die Effektivität dieser Spiele bisher nicht empirisch untersucht und verifiziert. Das Ziel dieser Arbeit besteht daher in der Analyse der Wirksamkeit und Effektivität der Spielelemente Abzeichen, Fortschrittsanzeige und Rangliste hinsichtlich Motivation und Leistung.

\section{Zielsetzung und Grenzen}
\todo[inline, color=green!40]{Als nächstes}

\section{Literatur- und Theorieüberblick}

\section{Zielsetzung und Fragestellung }

\subsection{Zielgruppe}

\subsection{Abgrenzung}

\section{Methoden}

\subsection{Studiendesign}

\section{Herausforderungen}

\section{Zeitplan}


\bibliography{expose}

\end{document}

%
% Please see the package documentation for more information
% on the APA6 document class:
%
% http://www.ctan.org/pkg/apa6
%